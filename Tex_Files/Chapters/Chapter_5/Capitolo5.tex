\chapter{Conclusioni}
Lo scopo dell’elaborato era di introdurre la visione di Internet of Things e
testare un possibile sviluppo applicativo tramite l'utilizzo della tecnologia  
LoRa ed il framework ESF sviluppato da Eurotech.\\
Dopo un'introduzione relativa agli ambiti applicativi e alle problematiche da
affrontare in questa visione, sono state introdotte le relative tecnologie
abilitanti per la realizzazione di  questo paradigma. 
Successivamente, si è parlato della modulazione LoRa e lo standard LoRaWAN
andandone a trattare gli aspetti principali che caratterizzano questa
tecnologia.
Dopo questa introduzione dello stato dell'arte, si è proseguito introducendo il
concetto di microservices parlando dei benefici  che una architettura basata su
microservizzi offre rispetto alla tradizionale architettura monolitica e come il
framework OSGi è stato concepito per poter implementare questo tipo di
architettura.
In seguito si è parlato dello sviluppo dei due bundle OSGi per
l'interfacciamento del digital baseband chip SX1301 con il framework sviluppato
da Eurotech e basato su OSGi ESF.
In fine si è voluto testare il range di operabilità dei moduli in esame per
verificare le reali potenzialità della tecnologi LoRa in un ambiente suburbano.
Pur avendo usato hardware adatto solo allo sviluppo \improvement{parola}  i
risultati ottenuti sono concordati a quelli dichiarati dalla casa produttrice
Semtech.
\section{Sviluppi futuri}
Come sviluppo futuro è necessario integrare un application server in grado di
cooperare con il framework ESF. Inoltre per ottenere le massime performance
dalla rete, è necessario creare un algoritmo ADR adattabile ai vari ambiti in
cui i devices opereranno.
