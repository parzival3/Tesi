\chapter{Conclusioni}
Lo scopo dell’elaborato era di introdurre la visione di Internet of Things e
testare un possibile sviluppo applicativo tramite l'utilizzo della tecnologia  
LoRa ed il framework ESF sviluppato da Eurotech.\\
Dopo un'introduzione relativa agli ambiti applicativi e alle problematiche da
affrontare in questa visione, sono state trattate le relative tecnologie
abilitanti per la realizzazione di  questo paradigma mettendone in evidenza i
vari pregi e difetti. 
Successivamente si è parlato della modulazione LoRa e lo standard LoRaWAN
andandone a trattare gli aspetti principali che caratterizzano questa
tecnologia.
Infine si è approfondito il concetto di microservizzi e di come questo concetto
è stato implementato nella piattaforma Everyware Software Framework di Eurotech.
Questa piattaforma si è dimostrata
particolarmente adatta a questa visione, poiché permette di risolvere alcuni
dei principali problemi legati allo sviluppo di applicazioni M2M, in particolare all’eterogeneità
dell’hardware relativa ai dispositivi in gioco. 
L’approccio con il framework OSGi risulta molto efficace poiché permette la
scrittura di  software di  ridotta complessità e dinamicamente aggiornabile, 
riuscendo così a ridurre il tempo di sviluppo e di conseguenza il time to market
dei nuovi prodotti.
Tramite questa piattaforma e l'utilizzo di due software aggiuntivi 
sono stati sviluppati due bundle , Mqtt Bridge Config e Lora Config,
per la gestione di un ricevitore LoRa, connesso al gateway prodotto da Eurotech ReliaGATE 10-11 
. A sviluppo completato è stata condotta una prova per testare la
distanza massima raggiungibile dal modulo radio in dotazione, la quale è
risultata coerente con i dati forniti dalla casa produttrice del ricevitore.
