\section{IoT}
Sempre più spesso si parla di Internet delle cose, o IOT, con tutto ciò si
intende un evoluzione delle applicazioni legate al settore mobile, al settore
della home automation e al settore embedded. Parliamo di oggetti i quali hanno
dei sensori e riescono a inviare informazioni nel cloud, andando poi ad
rielaborarle. 
Il punto principale di questo
upgrade sta nel riuscire a creare una rete di devices connessi ad internet. Per
supportare e utilizzare una potenza di calcolo maggiore andando a combinare
tecniche di data analytics per estrarre le informazioni più significative. 
In
questa visione, milioni di devices saranno connessi a Internet e molto presto
milioni di milioni di devices. 
Questo comporta delle sfide sia dal lato hardware sia dal lato software. Infatti
questi devices dovranno essere economici , avere una lunga durata della batterie
dal punto di vista hardware. Dal punto di vista software, dovranno essere sicuri
e dovranno essere facilmente controllabili e lavorare in maniera semi autonoma.
Oltre a ciò la  sfida è riuscire a creare delle sottoreti di devices relazionati tra di loro
riuscendo a creare un linguaggio comune per lo scambio delle informazioni ,
realizzando un sistema autonomo ed intelligente di sistemi intelligenti. Il
tutto si basa sull'utilizzo del cloud e l'analisi dei dati per far in modo di
migliorare la maggior parte dei settori, quali buisness, healt care, il
settore manifatturiero e quello delle energia per fare alcuni esempi, producendo
in maniera più veloce e con costi ridotti. 

cos'è una cosa , una cosa è qualsiasi tipo di oggetto il quale ha un sensore e
invia dati nel cloud in modo tale che quei dati possano essere analizzati e
riutilizzati , internet cellulari , tutto questo hanno cambiato il modo in cui
facciamo le cose, sia in modo personale sia nel mondo del bisnes  cosi anche
l'internet delle cose cambierà il nostro modo di fare le cose . Ci sono delle
sfide riguardo i devices che sono sia hardware che software come il consumo
della batteria e menagment, il fatto di non venire interrotti e compatibilità,
poi ci sono le sfide per creare sistemi che riescano a processare tutti i dati
genereati e riuscire a fare delle azioni intelligenti , problemi riguardo alla
sicurezza e privacy più ci muoviamo verso un modo sempre più connesso queste
difficolta diventano sempre più predominanti e difficili da risolvere

punti principali per i devices
\begin{itemize}
\item connettività
\item sicurezza
\item controllo
\end{itemize}
