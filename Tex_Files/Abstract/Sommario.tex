\begin{abstract}
Grazie al progresso dell’elettronica, si prevede che la presenza di dispositivi 
connessi, secondo il paradigma dell’Internet
delle Cose (IoT), aumenterà sostanzialmente nell’immediato futuro. 
Le dimensioni ridotte dei
dispositivi in commercio, come sensori, attuatori, tag e tanto altro, sono
particolarmente adatte a nuovi scenari applicativi.
Internet of Things
è la naturale evoluzione di Internet, ed è destinato a cambiare radicalmente
la nostra vita futura.Diversi standard sono attualmente in competizione per aggiudicarsi la
maggioranza del mercato e fornire la connettività su larga scala che è
richiesta da questi dispositivi. Tra questi standard, le Low Power Wide Area Networks (LPWAN) sono in
forte crescita, soprattutto grazie alla loro connettività a lungo raggio
sfruttando bande di frequenza libere. Questa Tesi si focalizzerà su una delle
tecnologie LPWAN predominanti: LoRa\tm e l'integrazione di questa tecnologia con
il framework ESF (Everyware Software Framework) sviluppato da Eurotech.
Prima di tutto verrà  introdotta la tecnologia LoRa ed il protocollo LoRaWAN.
Successivamente, verranno presentati due nuovi bundle installabili nel framework
ESF, tramite i quali sarà possibile controllare da remoto il comportamento dei
dispositivi LoRa.
\end{abstract}
