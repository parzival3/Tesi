\title{A Very Simple \LaTeXe{} Template}
\author{
        Vitaly Surazhsky \\
                Department of Computer Science\\
        Technion---Israel Institute of Technology\\
        Technion City, Haifa 32000, \underline{Israel}
            \and
        Yossi Gil\\
        Department of Computer Science\\
        Technion---Israel Institute of Technology\\
        Technion City, Haifa 32000, \underline{Israel}
}
\date{\today}

\documentclass[12pt]{article}

\begin{document}
\maketitle

\begin{abstract}
This is the paper's abstract \ldots
\end{abstract}

\section{Prima slide}
This is time for all good men to come to the aid of their party!

\section{Seconda slide}
\section{5 slide}
All'interno del paradigma dell'IoT è possibile individuare 3 livelli, il primo
chiamato device layer, è il livello base, formato da tutti i devices che generano i dati.
Il secondo livello è fomrato dal network layer, il quale rappresenta il mezzo di
comunicazione utilizzato dai devices per scambare le informazioni con il layer
supueriore detto application layer. Questo layer è composto da tutte le
applicazioni che gestiscono i devices ed i dati da essi ricevuti, questo livello
è il livello più importante, è qui che andremo ad estrapolare l'imformazione
persente all'interno di essi generando così il profitto e migliorando i servizi
collegati ai nostri dispositivi.
\section{6 slide}
Aspettandoci un così alto numero di devices sorgno spontanee alcune domande..
La principale è come riuscire a connettere tutti questi dispositivi
contemporaneamente. 
Attualmente le tecnologie wireless tradizionali non sono in grado di sopperire 
alle problematiche che l'IoT introduce  
\begin{itemize}
	\item	\textbf{Scalabilità} è necessario che la rete alla base di questi devices
			sia in grado di adattarsi in modo dinamico al numero di dispositivi
			presenti.
	\item	\textbf{Sicurezza} quando si parla di IoT è dobbligo parlare di
			sicurezza, anche perchè i devices dovranno avere una. 
	\item	\textbf{Durata della batteria} i dispositivi dovranno avere una
			durata della batteria pari a circa 10 anni.
	\item	\textbf{Raggio d'azione} la distanza con la quale questi devices
			devono raggiungere deve essere pari ad una 10 di chilometri.
	\item	\textbf{Costo} il prezzo del singolo devices deve essere basso.
\end{itemize}
Questo ha portato a ridisegnare il network layer.
\section{7 slide}
Con LPWAN o Low power wide area network si intendono tutte quelle tecnologie
create appositaemente per l'iot. Come dice il nome, queste tecnologie sono nate
per avere un basso consumo energetico, 
\begin{itemize}
	\item basso consumo energetico
	\item poter connettere un elevato numero di dispositivi disposti in una
		grande area geografica parliamo di alcune 10 di chilometri
\end{itemize}
Tutto ciò a discapito della data-rate raggiungibile dai nostri devices, ma
questo non è un problema poichè le informazioni che questi dispositivi
raccolgono sono piccole e quindi non è necessario disporre di una connessione
veloce e a lagra banda.
\section{8 slide}
Una delle LPWAN più interessanti è lora. lora nasce dalla Francese Semtech ed è
composta da due parti, una che rappresenta la modulazione con la quale i dati
vengono inviati, chiamata appunto LoRa e l'altra e LoRaWAN che rappresenta il
protocollo con il quale questi dati vengono scambiati. LoRa il layer fisico è
basato su frequenze ISM Industrial Scentific Medical le quali permetto uno
sviluppo molto veloce della rete rispetto ad tecnologie basate su frequenze
licenziate come ad esempio la rete cellulare. Questa tecnologia ha una copertura 
dichiarata da Semtech pari ad una decina di chilometri ed e studiata per essere 
resistente alle interferenze riguardanti altre comunicazioni e riguardanti gli
ostacoli come ad esempio pallazzi condomini ecc...  È così possibile riuscire a
garantire una copertura di alcuni chilometri anche in aree urbane.
Dall'altra parte LoRaWAN è un protocollo opensurce il quale definisce tre
tipologie di funzionamento riguardanti il devices e inoltre definisce la
topologia della rete, la quale è una topologi a stella.
\section{9 slide}
Una topologia di rete a stella è così formata, alla base ci sono i devices che
inviano il messaggio tramite la modulazione LoRa a dei gateway. Questi gateway
ricevuto il messaggio lora andranno ad inoltralo tramite una comunicazione
ethernet 4G 3G all'application layer. I devices IoT sono studiati principalmente
per spendere la maggior parte della loro vita in modalità deepsleep e risvegliarsi solo
all'occorrere di un evento o alla attivazione di un interupt.
Quando il messaggio viene inviato dal dispositivio, esso può essere ricevuto da
uno o più gatewys. Starà quindi all'application layer eliminare i duplicati e
scegliere il gateway più vicino per rispondere al dispositivo.
\section{10 slide}
Un altra problematica del mondo dell'IoT riguarda la architettura tarmite la
quale vengono progettate le applicazioni che controllono questi dispositivi.
Data la grande possibilità di scenari applicativi è necessario andare a
riprogettare le applicazioni tradizionali basate su di una struttura monolitica
andando a implementare quella che viene chiamata una struttura a microservizzi.



\section{Conclusions}\label{conclusions}
We worked hard, and achieved very little.

\bibliographystyle{abbrv}
\bibliography{simple}

\end{document}
This is never printed
